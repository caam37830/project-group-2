\documentclass{article}
\usepackage{verbatim}
\usepackage{amsmath}
\usepackage[margin=1in,left=1.5in,includefoot]{geometry}
\usepackage{graphicx}
\usepackage{systeme}
\usepackage{listings}
\usepackage{color}
\usepackage{xcolor}
\usepackage{graphicx}
\usepackage{biblatex}
\usepackage{float}
\usepackage{hyperref}
\usepackage{placeins}
\usepackage{fancyhdr}
\definecolor{mygreen}{RGB}{28,172,0} % color values Red, Green, Blue
\definecolor{mylilas}{RGB}{170,55,241}
\pagestyle{fancy}
\fancyhead{}
\fancyfoot{}
\fancyfoot[R]{ \thepage\ }
\usepackage[utf8]{inputenc}

% Default fixed font does not support bold face
\DeclareFixedFont{\ttb}{T1}{txtt}{bx}{n}{12} % for bold
\DeclareFixedFont{\ttm}{T1}{txtt}{m}{n}{12}  % for normal

% Custom colors
\usepackage{color}
\definecolor{deepblue}{rgb}{0,0,0.5}
\definecolor{deepred}{rgb}{0.6,0,0}
\definecolor{deepgreen}{rgb}{0,0.5,0}

\begin{document}


\begin{titlepage}
  \begin{center}

    \huge
    \textbf{Final Report}\\
   % \vfill
    \vspace{4.5cm}
    \LARGE
    Group 2: Yujin Liu, Helen Jin, Jiangmeng Li\\
    \vspace{4.5cm}
    \Large
    CAAM37830\\
    \vspace{1cm}
    University of Chicago\\
    \vspace{1cm}
    December 9, 2020

  \end{center}

\end{titlepage}

\section{Spatial Model and Variations}
\subsection{spatial agentbased model}
Firstly, we briefly introduce the agent based model in midterm checkpoint. We assumed that each person contact b people each day. For each infected person, the recover rate is k. We simulated the process of interaction for population and found the region in which all people would be infected. We also attained the phase transition boundary $b = 10*k - 3$.\\

Now we would like to introduce the spatial agent based model. This is how we implement the model. We introduce another Person() class with the pos attribute. For each individual, it will move and interact with other individuals according to movpos function. We also define the resetcorner and resetcenter functions to set the position of an individual so that we can implement the simulation function in different occasions.\\

This is the simulated model as we start all the individuals randomly. we set $k = 0.6$, $q = 0.08$, $p = 0.5$. We can see all the people will  finally be infected which is resonable since each individual will move with a large step size and radius q(0.08) is large as the population size is 20000. We can observe the case that all the susceptible people are quickly infected.\\

In this part, we would like to explore the relationship between the total infected individuals and p as infected people start fro,m center. We choose the parameter b = 0.5, k = 2 from phase transition boundary.
We can see that as p is increasing, the percentage of total infected people will increase to peak value at around 0.8 as p = 0.1. Then, it will sustain stable as p changes from 0.1 to 0.2. As p increses from 0.2, the percentage of total infected people will decrease, especially as $p > 0.7$.


The plot is reasonable. As p is small ($p=0.01$),an infected individual will not move, small amound of susceptible people will be infected.Initial infected people will be recovered and the total percentage of infected people will be small.


As $ 0.1<p<0.4 $, all the infected people will move around and contact most of susceptible people. The total percentage of infected people will be large. However, we choose the b,k from transition boundary and we can observe not all the people will finally be infected.


As p is large, some of infected people may stay at the original position because the step size is large. Thus, the total percentage of infected people will decrease. As p is extremely large, all the infected people will stay at the center. They can only infect the susceptible people around them initially. Only small percentage of population will finally be infected.\\


Now, we would like to discuss what will happen if all the people move from the corner or randomly spreaded. We choose the parameter p = 0.05
We can see that the percentage of total infected people will attain the maximum if all the infected people start from the random position in the grid. It will attain the minimum if all the infected people start from a single corner of the grid. This is reasonable. As all the infected individuals move from the random position, they will frequently interact with different people among the population and it will be stable with most of people are infected quickly.
However, if all the infected people start from a single corner, they will not move too far away from the single corner if the step is small. Thus, they will only contact with a small amount of susceptible people and all the infected people will quickly be recovered. We can say that total percentage of the infected people will be small if they start from a single corner and the step size is small.\\
Now we consider the case that all the people start from the center with a small step size p = 0.05. Since inital infected people will contact more people than the case that they start from the corner, the total percentage of infected people will be improved. However, the step size is small and all the infected people will also  quickly be recovered and the total percentage of infected individuals will be small than the case that all the infected people are random spreaded out in the grid.

\subsection{Spatial PDE Model}
Firstly , we briefly introduce the ode model in the midterm checkpoint. The implementation is quite straightforward. The input for the derivative are three variables s,i,r and a function for finding derivative is established. Then we run the ode simulation with the $solve_ivp$ function. The result is reasonable and we almost have the same transition phase boundary as what we explored in agent based model. In the following part, we use the parameter b =1 and k = 0.4. Not all the people will be susceptible in this case.

Now we would like to introduce how we implement the spatial pde model. We have several functions for generating the desired distribution of population. generatesum2 is the function for generating the population with initial infected people randomly spreaded out in the grid. generatesum2center is the function for setting the initial infected individuals at center while generatesum2corner is the function for setting the infected individuals at corner. We combine the s,i,r vector together as input for derivative function.We calculate the derivative accoring to the relationship:\\
$$\partial s(x,t)/\partial t = - b*s(x,t)*i(x,t) + p *L  s(x,t)$$
As for i(x,t) and r(x,t), similar relationship can be established. Since we have the function for finding the derivative, we can simulate the pde process. The following is the result.

The plot is reasonable. We choose the parameter b=1 and k = 0.4. We can observe the case that not all the individuals will finally be infected. We choose an appropriate diffusion term p = 1 so that all the susceptible people in the grids can be potentailly infected by the initial infected people located in ceter.

Now, we analyze how total percentage of infected people i changes with p. We choose the p in list [1,2,3,4,5,10,20,30,40,50] .We can see that as p is increasing, the total infected people will increase. The increment is not large.\\
 We can say that the percentage of total infected people will be large if diffusion term is large since all the people will interact with individuals located in different position frequently. The population will be stable quickly. \\
 We also find that the increment is not large as p is increasing. This is because we have the diffusion term p > =1 which guarantees the condition that the infected people can be spreaded out in the grid before most of the infected people in population are recovered.
 Thus,as p is increasing, we can observe the increment while it is not large.\\

 In this part, we try to observe what will happen if we choose the initial infected people from different location. As p = 20, the total infected percentage of population are 0.8926 ,0.88 ,0.86 as initial infected people are randomly spreaded out, located from center, located in a corner respectively. The rsult is reasonable. The total infected people will attain its maximum if the initial infected people are distrubted randomly in the grid because there is a chance that infected people may contact all the susceptible people in different positions before all the infected people are recovered.However, we control the relationship between b and k and ensure that not all the people will finally be infected.
 As for the case that all the infected people start from the center, we can see that the total percentage of infected people will be close to the first case since the diffusion term is large  and most of infected people will have a chance to contact susceptible people before they are recovered.As for the case that all the infected people initially located from a corner, the total percentage of infected people will be smaller because the infected people will contact less people due to the starting position before they are recovered.


 \section{Covid-19fit}
 In this variation, we will fit the Covid-19 dataset in China Hubei Province. Instead of using the SARS dataset claimed in midterm report, we find that Covid-19 dataset in China is complete. \\

Now, we would like to discuss how we fit the covid-19 data. We collect the dataset from ... in first 76 days. The covid-19 was controlled in China in around 76 days and building the specialized hospital helped the goverment to collect severe patients so that the transmission of covid-19 was controlled. We divide the process of simulation in two parts. 










\end{document}
